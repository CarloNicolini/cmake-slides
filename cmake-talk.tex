\documentclass[12pt] {beamer}

\usepackage[utf8] {inputenc}
\usepackage[english] {babel}
\usetheme {Air}
\usepackage {thumbpdf}
\usepackage {wasysym}
\usepackage {ucs}
\usepackage[utf8] {inputenc}
\usepackage {pgf,pgfarrows,pgfnodes,pgfautomata,pgfheaps,pgfshade}
\usepackage {verbatim}

\pdfinfo
{
    /Title       (Build facili con CMake)
    /Creator     (TeX)
    /Author      (Carlo Nicolini)
}


\title {Build facili con CMake}
\subtitle {Il sistema di build -definitivo-}
\author {Carlo Nicolini}
\date {\today}

\begin {document}

\frame {\titlepage}

\section* {}
\begin {frame}
\frametitle {Outline}
\tableofcontents[section=1,hidesubsections]
\end {frame}

\AtBeginSection[]
{
    \frame<handout:0>
    {
        \frametitle{Outline}
        \tableofcontents[currentsection,hideallsubsections]
    }
}

\AtBeginSubsection[]
{
    \frame<handout:0>
    {
        \frametitle{Outline}
        \tableofcontents[sectionstyle=show/hide,subsectionstyle=show/shaded/hide]
    }
}

\newcommand<> {\highlighton} [1] {%
\alt#2{\structure{#1}}{{#1}}
                                 }

\newcommand {\icon} [1] {\pgfimage[height=1em]{#1}}



%%%%%%%%%%%%%%%%%%%%%%%%%%%%%%%%%%%%%%%%%
%%%%%%%%%% Content starts here %%%%%%%%%%
%%%%%%%%%%%%%%%%%%%%%%%%%%%%%%%%%%%%%%%%%

\section{Introduzione}
\begin{frame}
\frametitle{Motivazioni}
\begin{itemize}
\item Split fra sviluppatore e maintainer
\item Portabilità su molti O.S.
\item Semplicità delle build
\item Migliore rapporto con cliente/comunit\'a di sviluppatori
\item Unit testing
\end{itemize}
\end{frame}

%%%%%%%%%%%%%%%%%%%%%%%%%%%%%%%%%%%%%%%%%%%%%%%%%%%%%%%%%%%%%%%%%%%%%%%%%%%%%%%%
\begin{frame}
\frametitle{Il passato}
\begin{block}{Sistema di build}
\begin{itemize}
\item Come fare una build?
\item Come differenziare Debug e Release
\item Flag al compilatore?
\item Come linkare? Shared o static?
\end{itemize}
\end{block}
\end{frame}

%%%%%%%%%%%%%%%%%%%%%%%%%%%%%%%%%%%%%%%%%%%%%%%%%%%%%%%%%%%%%%%%%%%%%%%%%%%%%%%%
\begin{frame}
\frametitle{In passato}
Compilazione manuale (anni 60):
\texttt{gcc -c mylibrary.cpp mylibrary.h -o mylibrary.o}

Sistemi di build automatici:

\begin{itemize}
\item Scons \textit{poco usato, cross-platform, necessita competenze di programmatore}
\item Autotools \textit{ molto usato, sintassi complicatissima }
\item Jam \textit{ buggy, dipendenze fatte a mano}
\item Waf \textit{lasciam perdere}
\item eccetera...
\end{itemize}
\end{frame}

%%%%%%%%%%%%%%%%%%%%%%%%%%%%%%%%%%%%%%%%%%%%%%%%%%%%%%%%%%%%%%%%%%%%%%%%%%%%%%%%
\begin{frame}
\frametitle{CMake}
Un sistema di build moderno gestisce building, testing e packaging tutto insieme.
\begin{itemize}
		\item Cross plattform (veramente).
		\item Dipendenze soddisfatte sempre (veramente).
		\item Un linguaggio di scripting che da libertà (vera).
		\begin{itemize}
			\item \texttt{\#define } a compile-time.
			\item Creazione menu Gnome, aggiunta icone e creazione setup personalizzati.
			\item Build differenziate per multi-architettura.
		\end{itemize}
		\item \textbf{Out-of-source} builds
\end{itemize}
\end{frame}

%%%%%%%%%%%%%%%%%%%%%%%%%%%%%%%%%%%%%%%%%%%%%%%%%%%%%%%%%%%%%%%%%%%%%%%%%%%%%%%%
\begin{frame}
\frametitle{CMake}
\begin{itemize}
	\item Sistema di \emph{meta-make}
	\item Genera Makefile o progetti (detti \emph{generatori})
	Kdevelop3, Eclipse, XCode, makefiles (Unix, NMake, Borland, Watcom, MinGW, MSYS, Cygwin), Code::Blocks etc
	\item Progetti multipli-eseguibili multipli
\end{itemize}
\end{frame}
%%%%%%%%%%%%%%%%%%%%%%%%%%%%%%%%%%%%%%%%%%%%%%%%%%%%%%%%%%%%%%%%%%%%%%%%%%%%%%%%
%\begin{frame}
%\frametitle{CMake workflow}
%	\begin{itemize}
%		\item CMakeLists.txt
%		\item Makefile/.vcproj/ .xcode
%		\item .obj/.o
%		\item .exe/ .dll/ .lib/ .dylib
%	\end{itemize}
%\end{frame}

%%%%%%%%%%%%%%%%%%%%%%%%%%%%%%%%%%%%%%%%%%%%%%%%%%%%%%%%%%%%%%%%%%%%%%%%%%%%%%%%
\begin{frame}
\frametitle{CMake tree e primi passi}
Supponiamo questo tree:

\begin{itemize}
\item \textbf{src}
\begin{itemize}
	\item myapp.cpp
	\item myapp.h
	\item CMakeLists.txt
\end{itemize}
\item \textbf{build}
\item CMakeLists.txt
\end{itemize}

Si fa la build con:
\begin{itemize}
	\item cmake .
	\item make
\end{itemize}
\end{frame}


%%%%%%%%%%%%%%%%%%%%%%%%%%%%%%%%%%%%%%%%%%%%%%%%%%%%%%%%%%%%%%%%%%%%%%%%%%%%%%%%

\section{Struttura di base di un CMakeLists.txt}

\begin{frame}
	\frametitle{HelloWorld eseguibile}
\begin{small}
\#Creiamo il nome del progetto

\textbf{PROJECT( helloworld )}

\# Impostiamo la variabile hello\_SRCS a contenere la hello.cpp

\textbf{SET( hello\_SRCS hello.cpp )}

\# Crea l'eseguibile di nome hello dal file contenuto nella variabile hello\_SRCS

\textbf{ADD\_EXECUTABLE( hello \$\{hello\_SRCS\} )}

\begin{itemize}
	\item Tutte le variabili sono \textbf{stringhe}
	\item Le variabili si dereferenziano bash-style \$\{NOMEVARIABILE\}
	\item Le variabili si impostano con SET
\end{itemize}
\end{small}
\end{frame}

%%%%%%%%%%%%%%%%%%%%%%%%%%%%%%%%%%%%%%%%%%%%%%%%%%%%%%%%%%%%%%%%%%%%%%%%%%%%%%%%

\begin{frame}
	\frametitle{Creazione libreria statica .a}
\begin{small}
\#Creiamo il nome del progetto

\textbf{PROJECT( mylibrary )}

\# Impostiamo la variabile mylibrary\_SRCS a contenere tutti i file che definiscono la libreria

\textbf{SET( mylibrary\_SRCS Foo.cpp Bar.cpp Qux.cpp)}

\# Crea una libreria STATICA (di default in CMake) a partire dai sorgenti

\# in Linux con gcc genera un file \emph{lib}myLibrary.a

\textbf{ADD\_LIBRARY( myLibrary \$\{mylibrary\_SRCS\} )}

\# Oppure crea una libreria SHARED (o DINAMICA in Windows) a partire dai sorgenti

\# in Linux con gcc genera un file \emph{lib}myLibrary.so
\textbf{ADD\_LIBRARY( myLibrary SHARED \$\{mylibrary\_SRCS\} )}
\end{small}
\end{frame}

%%%%%%%%%%%%%%%%%%%%%%%%%%%%%%%%%%%%%%%%%%%%%%%%%%%%%%%%%%%%%%%%%%%%%%%%%%%%%%%%%%%%%%%%%%%%%%%%

\begin{frame}
	\frametitle{Linguaggio - Variabili}
	\begin{itemize}
		\item Non serve dichiararle (stringa vuota se non esistono)
		\item Atipizzate
		\item SET crea e modifica variabili
		\item SET si affianca a LIST
		\item SEPARATE\_ARGUMENTS spezza argomenti separati da spazio in una LIST
		\item In Cmake 2.4: globali (name clashing problems)  In Cmake 2.6: scoped
		\item CMake corrente 2.9
\end{itemize}
\end{frame}

%%%%%%%%%%%%%%%%%%%%%%%%%%%%%%%%%%%%%%%%%%%%%%%%%%%%%%%%%%%%%%%%%%%%%%%%%%%%%%%%%%%%%%%%%%%%%%%%
\begin{frame}
	\frametitle{Linguaggio - Variabili}
	\begin{itemize}
		\item Non serve dichiararle (stringa vuota se non esistono)
		\item Atipizzate
		\item SET crea e modifica variabili
		\item SET si affianca a LIST
		\item SEPARATE\_ARGUMENTS spezza argomenti separati da spazio in una LIST
		\item In Cmake 2.4: globali (name clashing problems)  In Cmake 2.6: scoped
		\item CMake corrente 2.9
\end{itemize}
\end{frame}

%%%%%%%%%%%%%%%%%%%%%%%%%%%%%%%%%%%%%%%%%%%%%%%%%%%%%%%%%%%%%%%%%%%%%%%%%%%%%%%%%%%%%%%%%%%%%%%%

\begin{frame}
	\frametitle{Flessibilita - Costrutti condizionali}
	Variabili di sistema e costrutti
	
	\begin{small}
		if ( MSVC )
		
		endif ( MSVC )
		
		if ( UNIX )
		
		endif (UNIX)
		
		if (APPLE)
		
		endif(APPLE)
		
	\end{small}
	P.S. Tutte le variabili sono globali nel singolo CMakeLists.txt
	P.S. Non esiste un costrutto \textbf{switch}
\end{frame}

\begin{frame}
	\frametitle{Flessibilita - Espressioni regolari}
	Complicate ma possibili
	
	STRING( REGEX MATCH ... )
	
	STRING (REGEX MATCHALL ... )
	
	STRING(REGEX REPLACE ... )
	
\end{frame}

\begin{frame}
	\frametitle{Compatibilità con versioni precedenti}
	
	Molto importante impostare la compatibilita con le versioni precedenti
	
	Mantenere CMake up-to-date
	
	\textbf{CMAKE\_MINIMUM\_REQUIRED(VERSION 2.6.0 FATAL\_ERROR)}

\end{frame}

\begin{frame}
	\frametitle{Cmake cache}
	CMake salva le variabili non variate in un file CMakeCache.txt
	
	Velocissimo su Unix
	
	Lento su Windows (con compilatore MSVC)
\end{frame}

%%%%%%%%%%%%%%%%%%%%%%%%%%%%%%%%%%%%%%%%%%%%%%%%%%%%%%%%%%%%%%%%%%%%%%%%%%%%%%%%
%%%%%%%%%%%%%%%%%%%%%%%%%%%%%%%%%%%%%%%%%%%%%%%%%%%%%%%%%%%%%%%%%%%%%%%%%%%%%%%%
\section{Supporto a librerie esterne}
\begin{frame}
\frametitle{Highlighting}
\framesubtitle{Hey! Look here! here!}


\begin{alertblock}{If it's very very important...}
\alert{... you can "alert" in an "alertblock"}\\
Ewww, nasty, heh?
\end{alertblock}
\end{frame}

\section{Creazione pacchi con CPack}
\begin{frame}
\frametitle{Highlighting}
CPack è il progetto cugino di CMake con cui è strettamente integrato e permette di creare pacchi:
\begin{itemize}
\item Linux generici ( .sh, .tgz )
\item Distro-based ( .deb, .rpm, )
\item OSX (creazione .app o dischi .dmg )
\item Windows (creazione setup.exe grazie a NSIS installer e 7Zip )
\end{itemize}
\begin{itemize}
\item Pacchi architettura-specifici
\item Cross compilazione!
\end{itemize}

\end{frame}


\newcommand{\putlink} [1]{%
                          \pgfsetlinewidth{1.4pt}%
                          \pgfsetendarrow{\pgfarrowtriangle{4pt}}%
\pgfline{\pgfxy(1,1)}{\pgfxy(#1,1)}
                         }

\begin{frame}
\frametitle{Risorse online}
\begin{thebibliography}{10}

\beamertemplatearticlebibitems

\bibitem{CMake mailing list}
Mailing list ufficiale degli utenti di CMake
\newblock{\url{http://www.cmake.org/mailman/listinfo/cmake}}

               \bibitem{stackoverflow}
               Sito di domande e risposte, frequentato da molti utenti di CMake
               \newblock{\url{www.stackoverflow.com}}

               \bibitem{Cmake tutorial}
               CMake tutorial online
               \newblock{\url{http://www.cmake.org/cmake/help/cmake_tutorial.html}}

                              \bibitem{Mastering CMake}
                              Il libro ufficiale, insostituibile (in biblioteca di ingegneria a Mesiano )
                              \newblock{\url{http://www.kitware.com/products/books/CMakeBook.html}}

                                        \bibitem{kdeslides}
                                        Slides fatte con il tema beamer offerto da KDE
                                        \newblock {\tt http://www.kde.org/kdeslides/}

                                                \end{thebibliography}
                                                \end{frame}

\frame{
    \vspace{2cm}
    {\huge Domande ?}

    \vspace{3cm}
    \begin{flushright}
    Carlo Nicolini

    \structure{\footnotesize{nicolini.carlo@gmail.com}}
    \end{flushright}
}

\end{document}
