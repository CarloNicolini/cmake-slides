\documentclass[12pt]{beamer}

\usepackage[utf8]{inputenc}
\usepackage[english]{babel}
\usetheme{Air}
\usepackage{thumbpdf}
\usepackage{wasysym}
\usepackage{ucs}
\usepackage[utf8]{inputenc}
\usepackage{pgf,pgfarrows,pgfnodes,pgfautomata,pgfheaps,pgfshade}
\usepackage{verbatim}

\pdfinfo
{
  /Title       (Build facili con CMake)
  /Creator     (TeX)
  /Author      (Carlo Nicolini)
}


\title{Build facili con CMake}
\subtitle{Il sistema di build -definitivo-}
\author{Carlo Nicolini}
\date{\today}

\begin{document}

\frame{\titlepage}

\section*{}
\begin{frame}
  \frametitle{Outline}
  \tableofcontents[section=1,hidesubsections]
\end{frame}

\AtBeginSection[]
{
  \frame<handout:0>
  {
    \frametitle{Outline}
    \tableofcontents[currentsection,hideallsubsections]
  }
}

\AtBeginSubsection[]
{
  \frame<handout:0>
  {
    \frametitle{Outline}
    \tableofcontents[sectionstyle=show/hide,subsectionstyle=show/shaded/hide]
  }
}

\newcommand<>{\highlighton}[1]{%
  \alt#2{\structure{#1}}{{#1}}
}

\newcommand{\icon}[1]{\pgfimage[height=1em]{#1}}



%%%%%%%%%%%%%%%%%%%%%%%%%%%%%%%%%%%%%%%%%
%%%%%%%%%% Content starts here %%%%%%%%%%
%%%%%%%%%%%%%%%%%%%%%%%%%%%%%%%%%%%%%%%%%

\section{Introduzione}
\begin{frame}
  \frametitle{Qual'è il problema}

  \begin{block}{Beamer}
  \begin{itemize}
	\item Come fare una build?
	\item Come differenziare Debug e Release
    \item Cross-platform
  \end{itemize}
  \end{block}

\end{frame}

\section{Struttura di base di un CMakeLists.txt}
\begin{frame}
  \frametitle{Sections, Frames and Blocks}
  
  The current section is "Basic structuring". And the current frame
  is what you have on the screen right now.

  \begin{block}{A beautiful block}
  A block has a title, and some content. You can put in a block
  almost everything you want that is provided by LaTeX. For example
  math works as usual:
    \begin{equation}
    \sum_{i=1}^n i = \frac{n \times (n+1)}{2}
    \end{equation}
  \end{block}

  Also works outside a block:
  \begin{equation}
  \sum_{i=1}^n i^2 = \frac{n \times (n+1) \times (2n+1)}{6}
  \end{equation}
\end{frame}

\begin{frame}
  \frametitle{Different type of blocks}
  \framesubtitle{Weeeee! Colors!!}
  \begin{block}{Standard block}
  \begin{itemize}
    \item A standard block, used for grouping
    \item Obviously can contain itemizes too...
    \begin{itemize}
      \item And nested itemizes...
      \item of course!
    \end{itemize}
  \end{itemize}
  \end{block}
  \begin{alertblock}{Alert block}
  WARNING: Something very important inside this block!
  \end{alertblock}
  \begin{example}
  Note that examples are displayed as a special block...
  \end{example}
\end{frame}

\section{Supporto a librerie esterne}
\begin{frame}
  \frametitle{Highlighting}
  \framesubtitle{Hey! Look here! here!}


  \begin{alertblock}{If it's very very important...}
  \alert{... you can "alert" in an "alertblock"}\\
  Ewww, nasty, heh?
  \end{alertblock}
\end{frame}

\section{Creazione pacchi con CPack}
\begin{frame}
  \frametitle{Highlighting}
CPack è il progetto cugino di CMake con cui è strettamente integrato e permette di creare pacchi:
\begin{itemize}
	\item Linux generici ( .sh, .tgz )
	\item Distro-based ( .deb, .rpm, )
	\item OSX (creazione .app o dischi .dmg )
	\item Windows (creazione setup.exe grazie a NSIS installer e 7Zip )
\end{itemize}
\begin{itemize}
	\item Pacchi architettura-specifici
	\item Cross compilazione!
\end{itemize}

\end{frame}


\newcommand{\putlink}[1]{%
   \pgfsetlinewidth{1.4pt}%
   \pgfsetendarrow{\pgfarrowtriangle{4pt}}%
   \pgfline{\pgfxy(1,1)}{\pgfxy(#1,1)}
}

\begin{frame}
  \frametitle{Risorse online}
  \begin{thebibliography}{10}

  \beamertemplatearticlebibitems

	\bibitem{CMake mailing list}
	Mailing list ufficiale degli utenti di CMake
	\newblock{\url{http://www.cmake.org/mailman/listinfo/cmake}}
	
	\bibitem{stackoverflow}
	Sito di domande e risposte, frequentato da molti utenti di CMake
	\newblock{\url{www.stackoverflow.com}}
	
	\bibitem{Cmake tutorial}
	CMake tutorial online
	\newblock{\url{http://www.cmake.org/cmake/help/cmake_tutorial.html}}
	
	\bibitem{Mastering CMake}
	Il libro ufficiale, insostituibile (in biblioteca di ingegneria a Mesiano )
	\newblock{\url{http://www.kitware.com/products/books/CMakeBook.html}}
	
  \bibitem{kdeslides}
    Slides fatte con il tema beamer offerto da KDE
    \newblock {\tt http://www.kde.org/kdeslides/}

  \end{thebibliography}
\end{frame}

\frame{
  \vspace{2cm}
  {\huge Domande ?}
	
  \vspace{3cm}
  \begin{flushright}
    Carlo Nicolini

    \structure{\footnotesize{nicolini.carlo@gmail.com}}
  \end{flushright}
}

\end{document}
